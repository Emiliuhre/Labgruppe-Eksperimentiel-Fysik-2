\chapter{Discussion}
This experiment has a lot of glaring issues. First of all, we measure two different indices of refraction dependent on which medium the light is refracted into, when it according to theory should be constant.
This tells us that there must be something wrong with how we measure the angles. 
One obvious problem is the spreading of the light. 
This is a problem with our setup, as we must have used the lenses wrong. 
Other problems could be that the semicircle was not centerede correctly, 
such that the laser would move thorugh its center. 
We made sure to place the semicircle such that it was in the center, 
but it was not fixed on the rotation platform on which it rested, 
so it could potentially have moved.

There was also the issue that a significant portion of the light dissapeared somewhere, so that we could not calculate the reflection and transmission coeffitients using $V_0$. We are unsure as to why this is, but maybe the glass that we used absorbed alot of the light.

In spite of this, our model that the light decreased by a constant factor for both the reflected and transmitted light, seems to have worked great, as can be seen in figure \ref{fig:fresnell_comparison}. 

When would it be useful to understand the Fresnel relations? \newline
A real-world use would be using polarized sunglasses, when going fishing. 
With normal sunglass filters like neutral density filters of different types, you let both $s$- and $p$-polarized glare form the surface of the water pass to your eye.

A better option is to take advantage of the Fresnel relations to filter out the glare at angles $\le$ Brewster's angle \autocite{wiki:brewsters_angle}.
For an air-to-water interface, Brewster's angle is $\approx 56^\circ$ and vertically polarized sunglasses can block out most of the glare dependent on the angle of the sun.
With most of the $s$-polarized reflected light blocked from you eye, you can see below the surface of the water.
\begin{figure*}[h]
    \includegraphics[width=0.8\textwidth]{figures/Polarsied_vs_non_2_1024x1024.png}
        \centering
        \caption[Comparison of water reflection with and without polarized lenses.]{Comparison of water reflection with and without polarized lenses. Source: \autocite{luux_polarized_image}.}
        \label{fig:PolarizedFishing}
\end{figure*}