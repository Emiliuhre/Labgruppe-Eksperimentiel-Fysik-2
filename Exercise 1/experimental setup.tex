\chapter{Experimental Setup}

\subsection*{$s$- and $p$-polarization}

    To understand the Fresnel relations it is usefull to understand two linear polarization-components that behave different.\\
    The $s$ and $p$ direction is defined from the Plane of Incidence (POI) that is spanned by the incomming and reflected/transmitted lightbeam.
    \begin{itemize}
        \item $s$-polarized light is perpenducular to the POI, $s$ stands for senkrecht(German)
        \item $p$-polarized light is parallel to POI
    \end{itemize}

\subsection*{Refractive index $n$}
    To find the refractive index of the semicircle glass we use Snell's law
    \begin{align*}
        n_1 \sin(\theta_1) &= n_2 \sin(\theta_2) \tag{Snell's law}
        \intertext{If we choose $n_1$ to be the index of the glass and $n_2 \approx 1$ for air then,}
        \frac{n_1}{n_2} &= \frac{\sin(\theta_2)}{\theta_1}\\
        n_1 &= \frac{\sin(\theta_1+\theta_d)}{\theta_1}\\
    \end{align*}



\[
    r_s = \frac{n_1 \cos\theta_i - n_2 \cos\theta_t}{n_1 \cos\theta_i + n_2 \cos\theta_t}
\]

\[
    r_p = \frac{n_2 \cos\theta_i - n_1 \cos\theta_t}{n_2 \cos\theta_i + n_1 \cos\theta_t}
\]