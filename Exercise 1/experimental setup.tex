\chapter{Experimental Setup}
\section{Setup}
%
   \begin{figure}[h]
        \includegraphics[width=0.48\textwidth]{figures/EkspFys2_Ex1_setup_drawing_GlassToAir.png}
        \centering
        \caption{Experimental setup measuring transmitted light (Glass to air)}
        \label{fig:ExpsetupGlassToAir}
    \end{figure}
    
    \begin{figure}[h]
        \includegraphics[width=0.48\textwidth]{figures/EkspFys2_Ex1_setup_drawing_AirToGlass.png}
        \centering
        \caption{Experimental setup measuring transmitted light (Air to glass)}
        \label{fig:ExpsetupAirToGlass}
    \end{figure}

For the experiment we use a 650 nm diode laser\autocite{pasco_os8525a_manual}, such that the wavelength is somewhat constant. 
This laser is first sent through an aperture to decrease the intensity of the light, by obstructing some, because our sensor becomes saturated even at the lowest laser intensity.
Afterwards the beam is passed through a polarizer, set to 45 degrees, such that the light is equally polarized in the $p$- and $s$-direction.
Then the light passes through a semicircular piece of glass, either hitting the curved side, or the flat side first, but in both cases aimed such that the light passes the center of the circle.
The semicircular piece of glass is placed on a turntable, such that we can vary $\theta_i$. 
Hitting the center of the would-be circle ensures that the beam is normal to the curved side despite rotation of the semi-circle.
%
\begin{figure}[h]
        \includegraphics[width=0.48\textwidth]{figures/LaserSetup.jpg}
        \centering
        \caption{First part of the setup with the laser}
        \label{fig:LaserSetup}
    \end{figure}
%    
    \begin{figure}[h]
        \includegraphics[width=0.48\textwidth]{figures/SensorSetup.jpg}
        \centering
        \caption{Second part of the setup with the sensor}
        \label{fig:SensorSetup}
    \end{figure}
%
To measure the transmitted and reflected light we have a light sensor\autocite{pasco_ci6604_manual}, that is also mounted on a turn table, such that it can be turned to catch the light at different angles.
Both turntables have an angle scale, such that we can accurately measure the different angles. Here we do not just measure $\theta_i$ and $\theta_t$, but angles from which they can be derived.
Before the light enters the sensor it first goes through a collimating lens, such that we can catch light that spreads out and focus it on the sensor. Finally it goes through another polarizor, set such that only $p$- or $s$-polarized light respectively can get through.
%
\section{Measurements}
Measurements of the laser intensity are taken by a photodiode detector plugged into a PicoScope\autocite{PicoScope3203D}. 
The PicoScope measures the voltage over 1 second and calculates the mean voltage measured, together with a standard deviation.  
The measured voltage in the photodiode detector is proportional to the laser intensity. 
The angles are measured manually by reading on the angle scale. We measure 3 angles, $\theta_i$, $\theta_d = \theta_t \pm \theta_i$, and $\theta_R = 180^\circ - \theta_r - \theta_1$, where $\theta_r$ here is the reflection angle (where we expect $\theta_i = \theta_r$).
%
\section{Execution}
First the intensity of the laser is measured without the glass, for both $s$- and $p$-polarisation, as well as the background intensity.
Then the glass is placed, such that the laser travels through its center, and we measure the two mean intensities, standard errors and 3 angles. 
We then wary the angle of incidence, starting at 0 degrees, and moving in 5 degree increments up to 80 degrees. For each angle, we measure both the refracted and reflected beam, for both $s$- and $p$-polarized light.
We repeat this experiment twice, one where the light first hits the curved side, and one where it first hits the flat side.