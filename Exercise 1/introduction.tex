\chapter{Introduction}
This report will discuss the reflection and refraction of a beam of light at the surface of a transparent dielectric. 
This will be achived by using Snell's law to determine the index of refraction of a material, and then use this to attempt to prove the Fresnel Relations.
    \section{Snell's law}
        Snell's law descripe the relation between angels of incidence and refraction for light passing through different isotropic media. The law is given by
        \begin{align*}
            n_1 \sin(\theta_1) &= n_2 \sin(\theta_2) \tag{Snell's law}
        \end{align*}
        With refractive index $n_{\textup{1}}$ for the first media and angle of incidence $\theta_{\textup{1}}$. And for the refracted light in the second media $n_{\textup{2}}$ and $\theta_{\textup{2}}$
    \section{Refractive index $n$}
        To find the refractive index of the semicircle glass we use Snell's law
        If we choose $n_1$ to be the index of the glass and $n_2 \approx 1$ for air then;
        \begin{align*}
            \frac{n_1}{n_2} &= \frac{\sin(\theta_2)}{\theta_1}\\
            n_1 &= \frac{\sin(\theta_1+\theta_d)}{\theta_1}\\
        \end{align*}
    \section{$s$- and $p$-polarization}
        To understand the Fresnel relations it is usefull to understand two linear polarization-components that behave different.\\
        The $s$ and $p$ direction is defined from the Plane of Incidence (POI) that is spanned by the incomming and reflected/transmitted lightbeam.
        \begin{itemize}
            \item $s$-polarized light is perpenducular to the POI,\\ $s$ stands for senkrecht(German)
            \item $p$-polarized light is parallel to POI
        \end{itemize}
    
    \section{The Fresnel Relations}
        The Fresnel Relations describe the reflection and transmission of light on a plane of incidence of two different media. 
        There are two sets of Fresnel coefficients for two different linear polarization components.\\
        For s-polarized light 
        \[
            r_s = \frac{n_1 \cos\theta_i - n_2 \cos\theta_t}{n_1 \cos\theta_i + n_2 \cos\theta_t}
        \]
        \[
            t_s = \frac{2n_1 \cos\theta_i}{n_1 \cos\theta_i + n_2 \cos\theta_t}
        \]
        For p-polarized light
        \[
            r_p = \frac{n_2 \cos\theta_i - n_1 \cos\theta_t}{n_2 \cos\theta_i + n_1 \cos\theta_t}
        \]
        \[
            t_s = \frac{2n_1 \cos\theta_i}{n_2 \cos\theta_i + n_1 \cos\theta_t}
        \]
        From these relations we can derive the intensities of the reflected and refracted light by knowing that $R=r^2$ and $T=\frac{\cos{\theta_2}}{\cos{\theta_1}}\frac{n_2}{n_1}\:t^2$.
        \begin{align*}
            R_{\textup{p}} &= \frac{\tan^{2}(\theta_{\textup{1}} - \theta_{\textup{2}})}{\tan^{2}(\theta_{\textup{1}} + \theta_{\textup{2}})}\tag{1}\\
            R_{\textup{s}} &= \frac{\sin^{2}(\theta_{\textup{1}} - \theta_{\textup{2}})}{\sin^{2}(\theta_{\textup{1}} + \theta_{\textup{2}})}\tag{2}\\
            T_{\textup{p}} &= \frac{\sin{2\theta_{\textup{1}}}\sin{2\theta_{\textup{2}}}}{\sin^{2}(\theta_{\textup{1}} + \theta_{\textup{2}})\cos^{2}(\theta_{\textup{1}} - \theta_{\textup{2}})}\tag{3}\\
            T_{\textup{s}} &= \frac{\sin{2\theta_{\textup{1}}}\sin{2\theta_{\textup{2}}}}{\sin^{2}(\theta_{\textup{1}} + \theta_{\textup{2}})}\tag{4}
            \intertext{Where reflection and transmission add up to one: $R+T = 1$}
        \end{align*}
